%! Author = Luca Letter

%--------- Format bibliography and footnotes ---------
\usepackage[utf8]{inputenc}
\usepackage[style=authoryear-ibid,backend=biber,natbib]{biblatex}
% Always cite in parenthesis
\renewcommand{\textcite}{\citep}%
%\renewcommand{\cite}{\footfullcite}%
\renewcommand{\cite}{\textcite}%
\usepackage[bottom]{footmisc} % Put the footnotes at the bottom of the page
\usepackage{chngcntr}
\counterwithout{footnote}{chapter}% Always count up the footnotes and do not reset at each chapter
% Redefine these fields to start on a new line
\DeclareFieldFormat{url}{\newline\mkbibacro{URL}\addcolon\space\url{#1}}
\DeclareFieldFormat{doi}{\newline\mkbibacro{DOI}\addcolon\space\url{#1}}
\DeclareFieldFormat{eprint:arxiv}{\newline\mkbibacro{arXiv}\addcolon\space\url{#1}}
% Suppress specific fields in footnotes
\AtEveryCitekey{\clearfield{url}\clearfield{doi}\clearfield{issn}\clearfield{archivePrefix}}

\newcommand{\printbibliographyandsitography}{
    \printbibliography[heading=bibintoc, nottype=online, title={Bibliography}]
    \printbibliography[heading=bibintoc, type=online, title={Sitography}]}%
%-----------------------------------------------------

\usepackage{array} % Required for better table formatting
\usepackage{multirow} % Required to merge multiple rows in a table
\usepackage{svg} % For images and graphics path
\usepackage{graphicx} % For images and graphics path
\usepackage[subfigure]{tocloft} % For customizing the TOC, with subfigure compatibility option
%! suppress = MultipleIncludes
\usepackage{subfigure} % To combine multiple images in one figure
\usepackage[svgnames]{xcolor}
\usepackage{hyperref} % Create links in the document
\definecolor{linkGray}{rgb}{0.33, 0.33, 0.33} % Define a custom dark gray color
\hypersetup{
    colorlinks=true,
    allcolors=DimGrey, % Color for internal links (like TOC and section references)
    urlcolor=SteelBlue,
    menucolor=black,
}
\usepackage{newtxtext} % Use a "Times New Roman"-like font
\usepackage[nc]{newtxmath} % Use ScholaX font for math
\usepackage{titlesec} % For customizing section titles
\usepackage[labelfont=bf, labelsep=space]{caption} % Allow changing the captions (e.g. for figures and tables)
\usepackage{booktabs} % For better looking tables
\usepackage{etoolbox} % For table's text customization
\usepackage{float} % For figures and tables (floating) placement
\usepackage{cleveref} % For correct captions (figure and table) references
\usepackage{epigraph} % For fancy quotations

%--------- Format page margins ---------
\usepackage{geometry}
\geometry{
    top=3cm,
    right=3cm,
    bottom=3cm,
    left=4cm,
    headheight=1.5cm,
    bindingoffset=0cm,
}
%---------------------------------------

%--------- Format chapter's title ---------
%! suppress = MissingLabel
\titleformat{\chapter}[display]
{\normalfont\fontsize{16}{18}\bfseries} % Format: font, size, and boldness
{\chaptertitlename\ \thechapter} % Label: "Chapter I"
{20pt} % Separation between label and title
{\Huge} % Before-code: font size for the title
%! suppress = MissingLabel
\titlespacing*{\chapter}{0pt}{0pt}{40pt} % this alters "before" spacing (the second length argument) to 0
%! suppress = MissingLabel
\titleformat{\chapter}[hang]
{\normalfont\fontsize{16}{18}\bfseries\centering} % Format: font, size, boldness, and centering
{\chaptertitlename\ \Roman{chapter}} % Label: "Chapter I"
{10pt} % Separation between label and title
{} % Before-code
%-----------------------------------------------------------------

%--------- Format the Table of Contents ---------
\newcommand{\printtableofcontents}{
    \begingroup
    \hypersetup{linkcolor=black}
    \tableofcontents
    \endgroup}%

\renewcommand{\contentsname}{\hfill\fontsize{16}{18}\bfseries Summary\hfill} % Format and center the TOC's title
\renewcommand{\cftaftertoctitle}{\hfill} % Center the TOC's title
\setlength{\cftbeforetoctitleskip}{5pt} % Spacing before TOC's title

\renewcommand{\cftchappresnum}{Chapter~} % Prepend the text Chapter before the chapter's number and title
\renewcommand{\cftchapaftersnum}{} % Append the text after the chapter's number
\setlength{\cftchapnumwidth}{1.8cm} % Adjust the width to accommodate "Chapter X"
%-------------------------------------------------

%--------- Format sections's title ---------
%! suppress = MissingLabel
\titleformat{\section}
{\normalfont\fontsize{14}{16}\bfseries} % Format: font style, size, and boldness
{\thesection} % Section number
{10pt} % Space between number and title
{} % Before-code
%-------------------------------------------

%--------- Format subsections's title ---------
%! suppress = MissingLabel
\titleformat{\subsection}
{\normalfont\fontsize{12}{16}\bfseries} % Format: font style, size, and boldness
{\thesubsection} % Section number
{10pt} % Space between number and title
{} % Before-code
%-------------------------------------------

%! suppress = EscapeHashOutsideCommand
\DeclareCaptionLabelFormat{boldlabel}{\textbf{#1 #2} -}
%--------- Format figures and the List of Figures ---------
\counterwithout{figure}{chapter} % count asccending without considering the chapters
\renewcommand{\figurename}{Figure n.}
\captionsetup[figure]{labelformat=boldlabel, font=footnotesize, textfont=it}

\renewcommand{\cftloftitlefont}{\hfill\fontsize{16}{18}\bfseries} % Format and center the LoF's title
\renewcommand{\cftafterloftitle}{\hfill} % Center the LoF's title
\setlength{\cftbeforeloftitleskip}{5pt} % Spacing before LoF's title
%-------------------------------------------------

%--------- Format tables and the List of Tables ---------
\counterwithout{table}{chapter} % count asccending without considering the chapters
\renewcommand{\tablename}{Table n.}
\captionsetup[table]{labelformat=boldlabel, font=footnotesize, textfont=it}
\AtBeginEnvironment{tabular}{\footnotesize} % By default, use a smaller text in table

\renewcommand{\cftlottitlefont}{\hfill\fontsize{16}{18}\bfseries} % Format and center the LoT's title
\renewcommand{\cftafterlottitle}{\hfill} % Center the LoT's title
\setlength{\cftbeforelottitleskip}{5pt} % Spacing before LoT's title

% Define a new column type with a thicker vertical line
\newcolumntype{§}{!{\vrule width \heavyrulewidth}}
%---------------------------------------------

%--------- Glossary and other document appendices ---------
\newcommand{\printlistoffigures}{
    \clearpage
    \phantomsection
    \addcontentsline{toc}{chapter}{\listfigurename}
    \listoffigures}%
\newcommand{\printlistoftables}{
    \clearpage
    \phantomsection
    \addcontentsline{toc}{chapter}{\listtablename}
    \listoftables}%

\usepackage[acronym]{glossaries}
\newcommand{\printglossaryandacronyms}{
    \clearpage
    \phantomsection
    \addcontentsline{toc}{chapter}{\glossaryname}
    \printglossary
    \clearpage
    \phantomsection
    \addcontentsline{toc}{chapter}{Acronyms}
    \printglossary[type=\acronymtype]}%

\newcommand{\customchapter}[1]{ % Do not prepend text before the introduction title
    \clearpage
    \phantomsection
    \chapter*{#1}
    \label{ch:#1}
    \addcontentsline{toc}{chapter}{#1}
}

\newcommand{\beginappendix}{
    \appendix
    \customchapter{\appendixname}
    \renewcommand{\thesection}{\Alph{section}}
}
%----------------------------------------------------------

%--------- Format paragraph and text commands ---------
\setlength{\parindent}{0pt} % Removes indentation for all paragraphs
\newcommand{\textquote}[1]{\textquoteleft#1\textquoteright}%
\newcommand{\textdblquote}[1]{\textquotedblleft#1\textquotedblright}%

\usepackage{xstring}
%! suppress = UnresolvedReference
\newcommand{\tableref}[1]{%
    \IfBeginWith{#1}{tab:}%
    {%
        \hyperref[#1]{\tablename~\ref*{#1}}%
    }%
    {%
        \ClassError{uer_formatting}{Label `#1` is not a table (tab:) label}{}%
    }%
}
%! suppress = UnresolvedReference
\newcommand{\figureref}[1]{%
    \IfBeginWith{#1}{fig:}%
    {%
        \hyperref[#1]{\figurename~\ref*{#1}}%
    }%
    {%
        \ClassError{uer_formatting}{Label `#1` is not a figure (fig:) label}{}%
    }%
}
%! suppress = UnresolvedReference
\newcommand{\chapterref}[1]{%
    \IfBeginWith{#1}{ch:}%
    {%
        \chaptername~\ref*{#1}~\nameref{#1}%
    }%
    {%
        \ClassError{uer_formatting}{Label `#1` is not a chapter (ch:) label}{}%
    }%
}

%! suppress = UnresolvedReference
\newcommand{\sectionref}[1]{%
    \IfBeginWith{#1}{sec:}%
    {%
        \chaptername~\ref*{#1}~\nameref{#1}%
    }%
    {%
        \ClassError{uer_formatting}{Label `#1` is not a section (sec:) label}{}%
    }%
}
%------------------------------------------------------
